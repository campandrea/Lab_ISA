
%%%%%%%%%%%%%%%%%%%%%%%%%%%%%%%%%%%%%%%%%%%%%%%%%%%%%%%%%%%%%%%%%%%%%%%%%%%%
% This will help you in writing your homebook
% Remember that the character % is a comment in latex

% Divide the work you have done in each of the chapters used 
% during the lab lessons in a new chapter, as in the example below, 
% using a coherent title 


% For each chapter you can include :

%-----------------------------
% VHDL file, using the sintax:


	%\begin{listato}
	%\lstinputlisting{./exeMPLE/listato1.vhd}
	%\end{listato}

% the path to the file must be correct, obviously
% Should you have listings written in other languages the method is the 
% same, but the language set up must be changed using a different 
% setting for the command \lstset{language=VHDL} in file homebook.tex 


%-----------------------------
% figures in postcript (ps) or encapsulated postcript (eps)
% format, using the syntax:

%	\begin{figure}[h]
%	\centering
%	\includegraphics[width=9cm]{./cap1/figure1.eps}
%	\caption{Put a caption if you want (didascalia...:)))}
%	\label{put-a-label-for-referring-to-this-picture}	
%	\end{figure}

% the path to the file must be correct, obviously
% you can refer to this picture in any point of your document
% by typing the instruction:

% 	\ref{put-a-label-for-referring-to-this-picture}

% that is using the same label you put in the fiure label
% when you will run the "latex command" an automatic reference to
% this figure with the correct enumeration will be inserted


%-----------------------
% comment in text format (if you are not skilled in latex and don't want to be)
% using the sintax:

	%\begin{verbatim}
	% blablabla 
	%\end{verbatim}

% The verbatimg includes text as it is, as you could write in a normal text file 

% (BETTER) If uou want to write enhancing all the latex possibilities you 
% should add to you text a few commands in some particular cases. 
% In the following you have and example of a few chapters roughtly commented
% and written all in this file: remember that you can saparate
% each chapter in different files (this is always what a latex pro does) 
% and include them using the instruction: \input{./directoryxx/fileyy.tex}


%%%%%%%%%%%%%%%%%%%%%%%%%%%%%%%%%%%%%%%%%%%%%%%%%%%%%%%%%%%%%%%%%%%%%%%%%%%%%%%
%%%%%%%%%%%%%%%%%%%%%%%%%%%%%%%%%%%%%%%%%%%%%%%%%%%%%%%%%%%%%%%%%%%%%%%%%%%%%%%%%
%%%%%%%%%%%%%%%%%%%%%%%%%%%%%%
% Beginning of latex commands
% You can copy this in a new file (e.g. cap1/cap1.tex) and inlcude it here
% using the command : \input{./cap1/cap1.tex}



\chapter{Reference model development}


\chapter{VLSI implementation}


% include here only file for the third lesson and homeworks
% here the path to figures and VHDL should be ./exe3/

