\section{Synthesis}
For all the architectures described, synthesis was carried out with the Synopsys software in order to find the maximum operating frequency and the corresponding area. To find these results, the clock period was first set to zero, and then this value was increased until slack was null.

\subsection{Results}

The results obtained from the various syntheses are shown in \autoref{tab:syn_results}.

\begin{table}[h]
\begin{center}
\begin{tabular}{|l|l|l|}
\hline
Architercture & Minimum period (ns) & Area ($\SI{}{\micro\meter}^2$)\\
\hline
standard & 1.56 & 4047.7\\
\hline
carry save adder & 4.28 & 4851.6\\
\hline
parallel prefix adder & 1.45 & 4088.4\\
\hline
fine grain pipelining & 0.79 & 4924.7\\
\hline
fine grain pipelining (compile\_ultra) & 1.52 & 4188.2\\
\hline
modified Booth encoding & 1.94 & 5360.4\\
\hline
modified Booth encoding (compile\_ultra) & 1.64 & 5327.7\\
\hline
\end{tabular}
\end{center}
\caption{Syntesis Results}
\label{tab:syn_results}
\end{table}

\noindent A first observation that can be made is about the standard architecture: the results show values not so far from the other architectures, a sign that leaving freedom to the synthesiser is not always a bad choice.\\
The combined solution, on the other hand, in which the implementation of the adder, which is present in the multiplier of the second stage, is imposed, can lead to totally different and sometimes counter-productive results; in fact, by choosing a carry save adder the minimum period increases a lot, about 175\%, and with it the area, while choosing a parallel-prefix adder we have a slight improvement in frequency, about 8\%, keeping the area about the same as the standard implementation.\\
The solution with fine grain pipelining is the one with the smallest clock period. However, two completely different results are obtained: by forcing retiming (command \texttt{optimize\_register}) with consequent classical compilation, or by using only the command \texttt{compile\_ultra}, which enables various optimization mechanisms such as automatic ungrouping to optimize the timing of the circuit. In the first case a marked reduction in the clock period is observed, about 50\%, at the expense of an increase in the area of 22\%. In the second case there is only a 3\% reduction in the clock period and also a slight increase in the area, but less than in the previous case. It could then be concluded that, in this particular situation it is not convenient to use this technique unless there are strict constraints on the area to be occupied. On the other hand, it is very efficient in terms of performance to add registers and force the synthesiser to perform retiming.\\
Finally, implementing the second stage multiplier with the modified Booth Encoding technique, no special advantage is noticed, on the contrary, the clock period increases by 24\% and the area by 32\%, a result probably due to the higher combinatorial path for the addition of the Dadda tree. Slightly better results can be obtained by using the command \texttt{compile\_ultra}.
\medbreak
In conclusion, it can be said that to improve circuit performance the best strategy is to let the synthesiser choose the implementation of the multiplier and add a registers stage to give more margin in the application of retiming at the cost of paying in terms of area and latency. If this solution is not applicable, then implementing the multiplier with a parallel prefix adder brings better advantages than implementing it with a carry save adder or modified Booth encoding multiplier in terms of both area and maximum operating frequency.
