\subsection{Execution stage}
Lo stage di esecuzione racchiude i blocchi che si occupano di effettuare i calcoli tra cui la ALU, la Forwarding Unit, l'incrementatore del PC e alcune unità utili a gestire i flussi dei dati. Ogni operazione dura un colpo di clock quindi non sono presenti stadi di pipeline all'interno dello stage ma solo prima e dopo. In figura \todo{Inserire figura} si possono notare tutti i blocchi e il percorso dei dati. La ALU ha due ingressi i quali sono in parte gestiti dall'istruzione decodificata dalla CU e in parte dalla forwarding unit nel caso in cui siano presenti delle data dependencies. In caso di funzionamento senza hazard, sulla porta "A" della ALU è presente il dato "A" proveniente dal register file oppure il PC non incrementato, mentre sulla porta "B" è presente il dato proveniente dalla porta "B" del register file oppure l'Immediate ricavato dall'istruzione. In questo modo possono essere gestite tutte le istruzioni senza mai avere overlap nella richiesat delle porte. Quando invece sono presenti data dependencies, questi dati vengono ignorati e bypassati dalla decisione della forwarding unit al fine di inserire il dato corretto. La gestione dell'istruzione di branch richiede la valutazione dell'uguaglianza tra due dati e viene effettuata da un'unità separata dalla ALU in quanto occupata a calcolare l'indirizzo di salto ovvero la somma tra il PC e il displacement presente nell'Immediate. Nelle successive sottosezioni vengono analizzate nel dettaglio alcune unità.

\subsubsection{ALU}

\subsubsection{Forwarding Unit}
