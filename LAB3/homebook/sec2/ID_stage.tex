\subsection{Instruction Decode stage}
The decode stage contains the blocks used to generate the control and data signals from the fetched instruction. The control unit decodes the instruction and generates the control signals for the other functional units. The register file extracts the address fields from the instruction and asynchronously outputs the words associated with the addressed cells. In this stage there is also the immediate generation unit which, starting from the data in the instruction, makes the sign extension and generates the immediate data on 32 bits that can be used in the execution stage. Finally, another main block is dedicated to hazard detection for jump or load instructions.

\subsubsection{Control Unit}

\subsubsection{Hazard Detection Unit}
As described in the \autoref{subs:data_dep}, the Hazard detection unit is primarily concerned with resolving control hazards by inserting NOPs into the pipeline. 
When the current instruction is of branch type, it checks the output signal of the Branch Comparator so that when a jump occurs, it commands the IF and ID registers to flush and selects the jump address calculated by the ALU to be loaded into the PC.\\
It also handles load hazards in the same way, i.e. those data dependencies that cannot be resolved by data forwarding. One case is when a load instruction is followed by an operation that refers to data taken from memory. It can only be solved at compile time by changing the instruction order.

\subsubsection{Immediate generation}
