\subsection{Instruction Decode stage}
The decode stage contains the blocks used to generate the control and data signals from the fetched instruction. The control unit decodes the instruction and generates the control signals for the other functional units. The register file extracts the address fields from the instruction and asynchronously outputs the words associated with the addressed cells. In this stage there is also the immediate generation unit which, starting from the data in the instruction, makes the sign extension and generates the immediate data on 32 bits that can be used in the execution stage. Finally, another main block is dedicated to hazard detection for jump or load instructions.

\subsubsection{Control Unit}
For the design of the control unit, a fully combinatorial circuit similar to a decoder has been designed. The only input signal is the instruction from the fetch stage, while several control signals are generated at the output. The seven least significant bits, which always contain the instruction \textit{opcode}, are analysed, and from this all the control signals that manage the blocks of all the stages are generated. The signals that propagate to blocks located at different points in the pipeline must be properly delayed in order to synchronise the controls with the data.\\
The signals generated by the control unit relate to the commands for reading and writing the data memory, writing the register file and choosing the data to be written; two control flags are managed to report a possible branch or jump instruction. The commands for selecting the data and the type of operation to be performed by the ALU are also generated. Finally, a control signal for the immediate generation block is properly set.


\subsubsection{Hazard Detection Unit}
As described in the \autoref{subs:data_dep}, the Hazard detection unit is primarily concerned with resolving control hazards by inserting NOPs into the pipeline.
When the current instruction is of branch type, it checks the output signal of the Branch Comparator so that when a jump occurs, it commands the IF and ID registers to flush and selects the jump address calculated by the ALU to be loaded into the PC.\\
It also handles load hazards in the same way, i.e. those data dependencies that cannot be resolved by data forwarding. One case is when a load instruction is followed by an operation that refers to data taken from memory. It can only be solved at compile time by changing the instruction order.

\subsubsection{Immediate generation}
As described in the \autoref{subsection:ISA}, all instruction formats except R-type have a data element directly within the instruction called immediate. Each type of instruction has a different organization, so different operations must be performed for each format. The generation of the actual 32-bit data to be sent later to the execution stage is done by a decoder-like circuit that performs the following operations:
\begin{description}
    \item \textbf{I type and S type:} the data is represented on 12 bits and the sign extension is performed.
    \item \textbf{B type and J type:} the data is represented on 12 bits, and sign extension and left shift of one position is performed.
    \item \textbf{U type:} the data is represented on 20 bits and is left shifted by 12 positions.
\end{description}
