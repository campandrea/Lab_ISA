\subsection{Pipelined Architecture}
Mostrare e spiegare l'archiettura con pipeline senza entrare nel dettaglio degli hazard
Spiegare ogni stadio cosa fa

Partendo dall'architettura single cycle è possibile realizzare un'altra versione del processore RISC-V nella quale si inserisce la pipeline, si può quindi sfruttare l'instruction level parallelism per migliorare il throughput. L'esecuzione viene suddivsa in cinque stadi di pipe corrispondenti alle fasi nelle quali può essere suddivisa l'esecuzione di un'istruzione; così facendo possono essere eseguite più instruzioni nello stesso tempo e inoltre la latenza è la stessa per ogni tipo di istruzione.
Per poter avere un corretto funzionamento del circuito bisogna adottare particolari accorgimenti che possono complicare l'architettura. In prima battuta è necessario appicare la pipeline anche ai segnali di controllo insieme ai dati, in questo modo ad ogni stadio eseguirà le giuste operazioni sui dati presenti poichè ognuno di essi ad ogni colpo di clock opererà su differenti istruzioni.
Questo però non basta perche bisogna anche occuparsi dei vari tipi di hazard che si possono avere:
\begin{description}
\item \textbf{Structural hazard:} Due istruzioni hanno necessità della stessa risorsa nello stesso istante. Viene sempre evitato facendo un buon design.

\item \textbf{Data Hazard:} Dovuto alle data dependencies presenti tra istruzioni adiacenti.

\item \textbf{Control Hazard:} Quando il flusso di esecuzione dipende dalle istruzioni precedenti.
\end{description}
